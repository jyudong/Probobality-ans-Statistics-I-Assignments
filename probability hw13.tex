\documentclass[10.5pt]{article}
\usepackage{amsmath,amssymb,amsthm}
\usepackage{listings}
\usepackage{graphicx}
\usepackage[shortlabels]{enumitem}
\usepackage{tikz}
\usepackage[margin=1in]{geometry}
\usepackage{fancyhdr}
\usepackage{epsfig} %% for loading postscript figures
\usepackage{amsmath}
\usepackage{float}
\usepackage{amssymb}
\usepackage{caption}
\usepackage{subfigure}
\usepackage{graphics}
\usepackage{titlesec}
\usepackage{mathrsfs}
\usepackage{amsfonts}
\usepackage{indentfirst}
\usepackage{fancybox}
\usepackage{tikz}
\usepackage{algorithm}
\usepackage{algcompatible}
\usepackage{xeCJK}
\usepackage{extarrows}
\setCJKmainfont{Kai}

\title{PROBABILITY AND STATISTICS I
\\HOMEWORK XIII}
\author{\\Jianyu Dong   ~~2019511017}
\date{May, 29~ 2021}

\begin{document}
    
\maketitle
\newpage

\section{}
Since $(X_i)_{i=1}^n$ and $(Y_i)_{i=1}^m$ are random samples from $N(\mu_1,\sigma_1^2)$ and $N(\mu_2,\sigma_2^2)$ and the two samples are independent, we have that the sample means $\overline{X}_n$ follows $N(\mu_1,\frac{\sigma_1^2}{n})$ and $\overline{Y}_m$ follows $N(\mu_2,\frac{\sigma_2^2}{m})$. So that we have $\overline{X}_n-\overline{Y}_m$ follows $N(\mu_1-\mu_2,\frac{\sigma_1^2}{n}+\frac{\sigma_2^2}{m})$.\\\indent 
Since $\sigma_1=\sigma_2$, let $\sigma_1=\sigma_2=\sigma$. Thus, we have that $\overline{X}_n-\overline{Y}_n$ follows $N(\mu_1-\mu_2,\frac{\sigma^2}{n+m})$.\\\indent
Let the sample variance be $$S^2=\frac{n-1}{n+m-2}S_X^2+\frac{m-1}{n+m-2}S_Y^2$$\indent
Since all parameters are unknown, by the theorem, we have that $\frac{(\overline{X}_n-\overline{Y}_n)-(\mu_1-\mu_2)}{S\sqrt{1/m+1/n}}$ follows t-distribution with parameter n+m-2. So we could formulate that $$\mathbf{P}\left(-t_{\alpha/2,n+m-2}<\frac{(\overline{X}_n-\overline{Y}_m)-(\mu_1-\mu_2)}{S\sqrt{\frac{1}{n}+\frac{1}{m}}}<t_{\alpha/2,n+m-2}\right)=1-\alpha,$$\indent
where $t_{\alpha/2,n+m-2}$ is the 1-$\alpha/2$ quantile of the t distribution with degrees of freedom n+m-2. Rewriting the inequality, we obtain that $$\mathbf{P}\left((\overline{X}_n-\overline{Y}_m)-t_{\alpha/2,n+m-2}S\sqrt{\frac{1}{n}+\frac{1}{m}}<\mu_1-\mu_2<(\overline{X}_n-\overline{Y}_m)+t_{\alpha/2,n+m-2}S\sqrt{\frac{1}{n}+\frac{1}{m}}\right)=1-\alpha.$$\indent
Thus, the two-sided coefficient $1-\alpha$ confidence interval for $\mu_1-\mu_2$ is $$\left((\overline{X}_n-\overline{Y}_m)-t_{\alpha/2,n+m-2}S\sqrt{\frac{1}{n}+\frac{1}{m}},~(\overline{X}_n-\overline{Y}_m)+t_{\alpha/2,n+m-2}S\sqrt{\frac{1}{n}+\frac{1}{m}}\right).$$

\section{}
Since $(X_i)_{i=1}^n$ and $(Y_i)_{i=1}^m$ are random samples from $N(\mu_1,\sigma_1^2)$ and $N(\mu_2,\sigma_2^2)$ and the two samples are independent, we have that $\frac{S_X^2/S_Y^2}{\sigma_1^2/\sigma_2^2}$ follows F distribution with n-1 and m-1 degrees of freedom. So we could formulate that $$\mathbf{P}\left(F_0^{-1}(\alpha/2)<\frac{S_X^2/S_Y^2}{\sigma_1^2/\sigma_2^2}<F_0^{-1}(1-\alpha/2)\right)=1-\alpha,$$\indent
where $F_0$ is the c.d.f. of the F distribution with n-1 and m-1 degrees of freedom and $S_X^2,~S_Y^2$ are the sample variances of $(X_i)_{i=1}^n,~(Y_j)_{j=1}^m$,$$S_X^2=\sum_{i=1}^n(X_i-\overline{X}_n)^2,~S_Y^2=\sum_{j=1}^m(Y_j-\overline{Y}_m)^2.$$\\\indent
Rewriting the inequality, we have that $$\mathbf{P}\left(\frac{S_X^2/S_Y^2}{F_0^{-1}(1-\alpha/2)}<\frac{\sigma_1^2}{\sigma_2^2}<\frac{S_X^2/S_Y^2}{F_0^{-1}(\alpha/2)}\right).$$\indent
Thus, the two-sided coefficient $1-\alpha$ confidence interval for $\sigma_1^2/\sigma_2^2$ is $$\left(\frac{S_X^2/S_Y^2}{F_0^{-1}(1-\alpha/2)},~\frac{S_X^2/S_Y^2}{F_0^{-1}(\alpha/2)}\right).$$

\section{}
Since $(X_i)_{i=1}^n$ are random samples from $N(\mu,\sigma2)$ and $\sigma^2=10$, we have $\frac{\overline{X}-\mu}{\sigma/\sqrt{n}}$ follows N(0,1). According to the question, we have that $$\mathbf{P}\left(\overline{X}-\frac{1}{2}<\mu<\overline{X}+\frac{1}{2}\right)=0.954.$$\indent
Rewriting the inequality, we obtain that $$\mathbf{P}\left(-\frac{1}{2\sigma/\sqrt{n}}<\frac{\overline{X}-\mu}{\sigma/\sqrt{n}}<\frac{1}{2\sigma/\sqrt{n}}\right)=0.954.$$\indent
So that $$\Phi(\frac{1}{2\sigma/\sqrt{n}})=0.977.$$\indent
According to the table, we have that $$\frac{1}{2\sigma/\sqrt{n}}=2.00.$$\indent
Since $\sigma=\sqrt{10}$, we have that n is 160.

\section{}
Since $(X_i)_{i=1}^n$ are random samples from $\Gamma(3,\beta)$, by the theorem, we have that $Y_i=2\beta X_i$ follows $\chi^2(6)$. By the theorem, we have that $\sum_{i=1}^nY_i$ follows $\chi^2(6n)$. By the theorem, we have the mean of $\sum_{i=1}^nY_i$ is 6n. To obtain a coefficient 1-$\alpha$ confidence interval, we have that $$\mathbf{P}(-k_{6n,1-\alpha/2}<n\overline{Y}_n<k_{6n,\alpha/2})=1-\alpha,$$\indent
where $k_{6n,\alpha/2}$ is the value that $\mathbf{P}(n\overline{Y}_n>k_{6n,\alpha/2})=\alpha/2$ and $k_{6n,1-\alpha/2}$ is the value that $\mathbf{P}(n\overline{Y}_n>k_{6n,1-\alpha/2})=1-\alpha/2$.\\\indent
Rewriting the inequality, we obtain that $$\mathbf{P}\left(\frac{k_{6n,1-\alpha/2}}{2n\overline{X}_n}<\beta<\frac{k_{6n,\alpha/2}}{2n\overline{X}_n}\right)=1-\alpha,$$\indent
where $\overline{X}_n$ is the sample mean.\\\indent
Thus, the two-sided coefficient $1-\alpha$ confidence interval for $\beta$ is $$\left(\frac{k_{6n,1-\alpha/2}}{2n\overline{X}_n},~\frac{k_{6n,\alpha/2}}{2n\overline{X}_n}\right)$$

\section{}
Since $(X_i)_{i=1}^n$ are random samples from $N(\mu,16)$, we have that $\frac{\overline{X}-\mu}{4/\sqrt{n}}$ follows N(0,1). To obtain a coefficient 1-$\alpha$ confidence interval with length $\leqslant L$, we have that $$\mathbf{P}\left(\overline{X}_n-\frac{L}{2}<\mu<\overline{X}_n+\frac{L}{2}\right)=1-\alpha.$$\indent
Rewriting the inequality, we obtain that $$\mathbf{P}\left(-\frac{\sqrt{n}L}{8}<\frac{\overline{X}-\mu}{4/\sqrt{n}}<\frac{\sqrt{n}L}{8}\right)=1-\alpha.$$\indent
Let $z_{\alpha/2}$ be the value that $\mathbf{P}(X>z_{\alpha/2})=\alpha/2$, so that we get $$\frac{\sqrt{n}L}{8}=z_{\alpha/2}.$$\indent
Thus, the smallest n is $$n=\left\lceil \frac{64z_{\alpha/2}^2}{L^2}\right\rceil.$$

\section{}
Since X has a Possion distribution with mean $\theta$, we have the p.d.f. of X given $\theta$ is $$p(X=x;\theta)=\frac{e^{-\theta}\theta^x}{x!},~for ~x=0,1,2,\dots$$\indent
zero elsewhere.\\\indent
Since $(X_i)_{i=1}^{10}$ are independent and each $X_i$ follows Possion distribution with mean $\theta$, then we have $\sum_{i=1}^{10}X_i$ follows the Possion distribution with mean $10\theta$.\\\indent
We have that the sample mean $\overline{X}=\frac{1}{10}\sum_{i=1}^{10}X_i$ is an estimator of $\theta$. We could determine the c.d.f. of $\overline{X}$ is $$F_{\overline{X}}(\overline{X};\theta)=\mathbf{P}(10\overline{X}\leqslant 5)=\sum_{n=0}^{5}\frac{e^{-10\theta}(10\theta)^n}{n!}.$$\indent
Then we could determine that $$\frac{d}{d\theta}\mathbf{P}(10\overline{X}\leqslant 5)=\sum_{n=0}^510e^{-10\theta}\frac{(10\theta)^{n-1}}{(n-1)!}(1-\frac{10\theta}{n})<0.$$\indent
By the definition, there is 
$$\begin{aligned}
    F_{\overline{X}}(\overline{X}-;\underline{\theta})=\sum_{n=0}^{4}\frac{e^{-10\underline{\theta}}(10\underline{\theta})^n}{n!}=0.95\\
    F_{\overline{X}}(\overline{X};\overline{\theta})=\sum_{n=0}^{5}\frac{e^{-10\overline{\theta}}(10\overline{\theta})^n}{n!}=0.05
\end{aligned}$$\indent
which implies that $$\begin{aligned}
    e^{-10\underline{\theta}}\left(1+10\underline{\theta}+\frac{10^2}{2}\underline{\theta}^2+\frac{10^3}{6}\underline{\theta}^3+\frac{10^4}{24}\underline{\theta}^4\right)&=0.95\\
    e^{-10\overline{\theta}}\left(1+10\overline{\theta}+\frac{10^2}{2}\overline{\theta}^2+\frac{10^3}{6}\overline{\theta}^3+\frac{10^4}{24}\overline{\theta}^4+\frac{10^5}{120}\overline{\theta}^5\right)&=0.05
\end{aligned}$$\indent
Then we could determine that $$\underline{\theta}=0.1970,~\overline{\theta}=1.0513$$\indent
Thus, the confidence interval with coefficient of at least 0.9 is (0.1970,1.0513).

\section{}
Since we have the 95\% confidence interval for the parameter $\mu$ of Possion($\mu$) distribution is (2,3), there is $$\mathbf{P}(2<\mu<3)=95\%.$$\indent
Then we let $z=e^{-\mu}$, so we get that $\mu=-\log z$ which is strictly decreasing. Thus, there is $$\mathbf{P}(2<-\log z<3)=95\%.$$\indent
Rewriting the inequality, there is that $$\mathbf{P}(e^{-3}<z<e^{-2})=95\%.$$\indent
So that the 95\% confidence interval for $\mathbf{P}(X=0)=e^{-\mu}$ is $(e^{-3},e^{-2})$.

\section{}
\subsection{}
Let $\overline{X}_{16}=\frac{1}{16}\sum_{i=1}^{16}X_i$ be the sample mean and $S_{X}^2=\sum_{i=1}^{16}(X_i-\overline{X}_{16})^2$ be the sample variance.\\\indent
By the theorem, we have that $Y=\frac{\overline{X}_{16}-\mu}{S_X/\sqrt{16}}$ follows the t distribution with 15 degrees of freedom. So that we have $$\mathbf{P}\left(-t_{0.025,15}<\frac{\overline{X}_{16}-\mu}{S_X/\sqrt{16}}<t_{0.025,15}\right)=0.95$$\indent
where $t_{0.025,15}$ is the value that $\mathbf{P}(Y>t_{0.025,15})=0.025$\\\indent
Rewriting the inequality, we get that $$\mathbf{P}\left(\overline{X}_{16}-t_{0.025,15}\frac{S_X}{4}<\mu<\overline{X}_{16}+t_{0.025,15}\frac{S_X}{4}\right)=0.95.$$\indent
According to the question, we have that $$\mathbf{P}(1.6<\mu<7.8)=0.95.$$\indent
Thus, we could determine the mean of the dataset is $$\overline{X}_{16}=\frac{1.6+7.8}{2}=4.7.$$
\subsection{}
Searching for table we have that $t_{0.025,15}=2.131$. Then we have that $4.7-2.131\times \frac{S_X}{4}=1.6$. Thus, we get that $S_X=5.819$\\\indent
If we perfer to have 99\% confidence interval for $\mu$, by the theorem, we have that $$\mathbf{P}\left(-t_{0.005,15}<\frac{\overline{X}_{16}-\mu}{S_X/\sqrt{16}}<t_{0.005,15}\right)=0.99$$\indent
Rewriting the inequality, we have that $$\mathbf{P}\left(\overline{X}_{16}-t_{0.005,15}\frac{S_X}{4}<\mu<\overline{X}_{16}+t_{0.005,15}\frac{S_X}{4}\right)=0.99$$\indent
Searching for the table we have that $t_{0.005,15}=2.947$. Then we could determine that $$\mathbf{P}\left(0.413<\mu<8.987\right)=0.99$$\indent
Thus, the 99\% confidence interval for $\mu$ is (0.413,8.987).

\end{document}

$$\begin{aligned}   
    \underline{\theta}&=inf\{\theta:F_{\overline{X}}(\overline{X}-;\theta)\leqslant 0.95\}\\
    \overline{\theta}&=sup\{\theta:F_{\overline{X}}(\overline{x};\theta)\geqslant 0.05\}
\end{aligned}$$
Then we could determine that $\underline{\theta}=0.2,~\overline{\theta}=1$.\\\indent
Thus, the confidence interval with coefficient of at least 0.9 is (0.2,1).
